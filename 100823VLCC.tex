\documentclass{article}
\usepackage[a4paper, total={6in, 10in}]{geometry}
\begin{document}
\section*{Storage}
\subsection*{Types of Storage}
\begin{enumerate}
    \item DAS : Directly Accessed Storage
    \item NAS : Network Attached Storage
    \item SAN : Storage Area Network
\end{enumerate}
\subsection*{DAS}
Data storage system attached with host hard drive without any network (Personal Use, Less Cost, Low Maintenance)
\subsection*{NAS}
It is a computer data storage server connected to/with computer network and provides access to group of clients. (Good for working on small scale projects)
\subsection*{SAN}
It is a dedicated and independent high speed network that interconnects and delivers a shared pool of storage device to multiple servers. (Mainly used for large projects, costly and high maintenance)
\subsection*{Worst Case Scenario in SAN}
Budget is low, requirement is low, but still SAN is acquired/bought. If the staff was less in number, and budget was also low, then a SAN was not necessary.
\subsection*{Best Case Scenario in SAN}
Project was large scale and mission critical, and a large organization was involved, then acquiring a SAN was necessary.
\section*{Some advantages of SAN}
\begin{enumerate}
    \item Storage exists independent of application
    \item Better availability, reliability and serviceability
    \item Better application performance
    \item Centralized and Consolidated
    \item Remote site data transfer and vaulting: Remote copy of data protects from disasters
    \item Simple centralized management: Simplify management by creating single image
\end{enumerate}
\section*{VSAN (Virtual Storage Area Network)}
It is a storage based component that provides a virtualized pool. Communication is done via iSCSI or fibre channel.
\section*{Distributed Computing}
\subsection*{Distributed Systems}
It is a collection of independent entities that cooperate to solve a problem that cannot be individually solved. Distributed computing can be defined as a collection of multiple or autonomous computer systems linked by a computer network and being equipped by distributed systems.
\subsection*{Characteristics of Distributed Computing}
\begin{enumerate}
    \item Heterogeniety is hidden by the user
    \item Internal systems are not shared (like RAM)
    \item Availability must be present inspite of failure
    \item Information Sharing
    \item No common physical clock
    \item No shared memory
    \item Geographical Separation
    \item Processors are loosely coupled (please google what is the meaning of loosely coupled)
\end{enumerate} 
\subsection*{Intranet}
It is a small part of internet being used by one organization separately administered.
\section*{Parallel Computing}
It is the use of multiple processing elements simultaneously for solving any problem. It saves time and money. Serial computing wastes computer potential, thus parallel computing makes better use of hardware. It can take advantage of non-local resources when resources are finite. In it many executions/calculations of a program take place parallely.
\end{document}